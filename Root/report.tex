\documentclass[12pt]{ctexart}
\usepackage{geometry}       % 设置页面整体布局
\geometry{top=2.5cm, bottom=2.5cm, left=2cm, right=2cm}
\usepackage{fancyhdr}       % 设置页眉页脚布局
\pagestyle{fancy}
\rhead{\thepage}            % 设置右页眉为页数
\chead{中国科学技术大学}
\cfoot{}                    % 设置中间页脚为空
\usepackage{amsmath}        % 数学公式宏包
\numberwithin{equation}{section}
\usepackage{esint}          % 交叉引用宏包
\usepackage[colorlinks,     % 设置引用的颜色
            linkcolor=black,
            anchorcolor=black,
            urlcolor=cyan,
            citecolor=black,
           ]{hyperref}
\usepackage{makecell}       % 插入表格宏包
\usepackage{longtable}      % 长表格宏包
\usepackage{appendix}       % 生成附录宏包
\usepackage{graphicx}       % 插入图片宏包
\usepackage{epstopdf}       % 插入eps图片宏包
\usepackage{cite}           % 文献引用宏包
\renewcommand{\thefigure}   % 设置图片编号格式
    {\thesection{}.\arabic{figure}}
\renewcommand{\thefootnote}{} % 设置角标编号不出现在文中
                            % 以\footnotetext{Footnotetext without footnote mark}使用
\usepackage{unicode-math}
\usepackage{listings}
\usepackage{hyperref}



\CTEXsetup[format={\Large\bfseries}]{section}

\begin{document}

\nocite{*}

\begin{center}
    \heiti \fontsize{24pt}{0}{非线性方程求根}

    \vspace{12pt}

    \kaishu \fontsize{13.75pt}{0}禤科材
    

    \footnotetext{\textbf{实验日期:}2023年4月22日}
    \footnotetext{\textbf{作者简介:}禤科材(2002-),男,学号PB20030874,中国科学技术大学本科在读,专业方向为化学物理}
    \footnotetext{\textbf{联系方式:}电话 18108064415 ,邮箱 \href{mailto:ustcxkc@mail.ustc.edu.cn}{ustcxkc@mail.ustc.edu.cn}}

    \vspace{5pt}

    \songti \fontsize{12pt}{0}(中国科学技术大学化学与材料科学学院,安徽 合肥 230026)
\end{center}


\section{实验原理}

对于非线性方程求根问题,主要有两种方法:牛顿法与弦截法。

\subsection{牛顿法及其二次收敛性}
牛顿法的迭代公式为
\begin{equation}\label{equ:newton}
    x_{i+1} = x_i - \frac{f(x_i)}{f'(x_i)} 
\end{equation}
牛顿法是一个二次收敛的算法,证明如下。将 $f(x)$ 在 $r$ 处泰勒展开到二阶:
\[
    f(r) = f(x_i) + (r - x_i) f'(x_i) + \frac{(r - x_i)^2}{2} f''(\xi_i)
\]
由于 $r$ 是根,我们有
\begin{align*}
    0 &= f(x_i) + (r - x_i) f'(x_i) + \frac{(r - x_i)^2}{2} f''(\xi_i) \\
    - \frac{f(x_i)}{f'(x_i)}  &= r - x_i + \frac{(r - x_i)^2}{2} \frac{f''(\xi_i)}{f'(x_i)}  
\end{align*}
这里假设 $f'(x_i) \neq 0$ 。重新整理则有
\begin{align*}
    x_i - - \frac{f(x_i)}{f'(x_i)} - r &= \frac{(r - x_i)^2}{2} \frac{f''(\xi_i)}{f'(x_i)}  \\
    x_{i+1} - r = e_i^2 \frac{f''(\xi_i)}{2f'(x_i)}  \\
    e_{i+1} = e_i^2 \left|\frac{f''(\xi_i)}{2f'(x_i)} \right| 
\end{align*}
所以牛顿法是二次收敛的。

\subsection{弦截法的超线性收敛}
弦截法的迭代公式为
\begin{equation}\label{equ:secant}
    x_{i+1} = x_i - \frac{f(x_i)(x_i - x_{i-1})}{f(x_i) - f(x_{i-1})} 
\end{equation}
弦截法的收敛阶为 $1.618$ ,稍慢于牛顿法,证明如下。代数整理并舍弃高阶项,易得
\[
    e_{i+1} = \left|\frac{f''(\xi_i)}{2f'(x_i)} \right| e_ie_{i-1}
\]
假设它的收敛阶为 $\alpha$ ,则
\[
    e_{i+1} = \left|\frac{f''(\xi_i)}{2f'(x_i)} \right|^{\alpha -1} e_i^\alpha 
\]
再取极限,解得
\[
    \alpha = \frac{1 + \sqrt{5}}{2} 
\]
证明完毕。

\section{实验方法}
使用 jupyter notebook 编程求解。

\section{实验结果}
对于函数
\[
    f(x) = \frac{x^3}{3} - x
\]
分别按照题目要求使用不同的初值求解,结果如下:


\begin{longtable}{ccc}
    \caption{牛顿法} \\
        \hline
        初值 & 根 & 步数 \\
        \hline
        $0.1$ & $0.0$ & $2$\\
         $0.2$ & $0.0$ 	&  $3$\\
	    $0.9$ &  $-1.7320508075688774$ &  $6$\\
	    $9.0$ &  $1.7320508075688774 $&  $9$
\end{longtable}

\begin{longtable}{ccc}
    \caption{弦截法} \\
        \hline
        初值 & 根 & 步数 \\
        \hline
        $[-0.1, 0.1]$ 	&  $0.0$ 	&  $1$ \\
	    $[-0.2, 0.2]$ 	&  $0.0$ 	&  $1$ \\
	    $[-2.0, 0.9]$ 	&  $1.7320508075688772 $ 	&  $11$ \\
	    $[0.9, 9.0]$ 	&  $1.7320508075688772 $ 	&  $13$
\end{longtable}

经过对比,牛顿法如果在非常靠近根的初始值开始,它所需要的收敛步数将大于弦截法;而在初始值不那么靠近根的情况下,牛顿法的迭代步数将会小于弦截法。

\end{document}